%%% Local Variables:
%%% mode: latex
%%% TeX-master: "../doc"
%%% coding: utf-8
%%% End:
% !TEX TS-program = pdflatexmk
% !TEX encoding = UTF-8 Unicode
% !TEX root = ../doc.tex

Students and teachers have to organize their time for different tasks. Although some applications for this purpose already exist, most if not all are not suitable for use in an academic environment. Therefore, a concept and technological prototype for an academic time planning application had been designed in a bachelor thesis prior to this semester thesis. The online platform Toggl Track was suggested to be used for time tracking. The aim of this semester thesis was to implement the desired functionality based on that concept using the prototype as initial framework. The concept of mapping tracked time data to planned time data was realized. To help the development process tests and dummy data were written. The data is stored in the JSON format. Functionalities to create, import, export and delete planning data were added in the front-end of the application as well as the possibility to fetch tracked time data from Toggl Track. A status display indicates the last time data has been fetched. Next, the linking of time data and planning data was implemented. To visualize the current state of planned and tracked time a graphical overview using bar and line charts was created. With all the implemented functionality the current version of the software is a usable prototype. Possible improvements include enhanced usability, finer granulation of the visualization, cross-platform functionality as well as the integration of other time tracking platforms. 
