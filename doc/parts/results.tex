%%% Local Variables:
%%% mode: latex
%%% TeX-master: "../doc"
%%% coding: utf-8
%%% End:
% !TEX TS-program = pdflatexmk
% !TEX encoding = UTF-8 Unicode
% !TEX root = ../doc.tex

This section shows the results achieved during realization of the project.

\section{Application data persistence with LocalStorage}
To store application data (plan project data, data fetched from Toggl etc.), the approach described in chapter \ref{Local Storage} was applied. The data is hold by a data manager object which is stored in Local Storage and can be loaded from there using the interface provided by the LocalStorage library.

\section{Overview PlanProject vs TogglProject} \label{Graphical overview}
To display the difference between the plan projects and the corresponding Toggl project, three different charts are displayed in the charts tab.
\begin{itemize}
	\item Total overview
	\item Projects overview
	\item Project time range overview
\end{itemize}
The different charts are explained in more detail in the following subsections.

\subsection{Total Overview}
This bar chart shows the overall sum of the time planned for all plan projects, compared to the total tracked time of their associated Toggl projects. Additionally, a prediction of how much time will be needed according to the future plan entries is displayed. An example for two projects is shown in \ref{figure9}.
\begin{figure}[H]
	\centering
	\includegraphics[width=1.0\columnwidth]{TotalOverview}
	\caption{Total overview for two projects}
	\label{figure9}
\end{figure}
If all projects have been completed already, meaning there are no future plan entries for any project, then no prediction will be displayed. An example for one project is shown in \ref{figure10}.
\begin{figure}[H]
	\centering
	\includegraphics[width=1.0\columnwidth]{TotalOverview_FinishedProject}
	\caption{Total overview for one finished project}
	\label{figure10}
\end{figure}

\subsection{Projects Overview}
This bar chart is similar to the total overview, but the single projects are displayed separately. An other minor difference is that the prognosis is added to both the tracked time and the planned time, creating a prediction for both. An example for two projects is shown in \ref{figure10}.
\begin{figure}[H]
	\centering
	\includegraphics[width=1.0\columnwidth]{ProjectOverview}
	\caption{Project overview for two projects}
	\label{figure11}
\end{figure}
If a project has been completed already, meaning there are no future plan entries for it, then no prediction will be displayed for that project. An example for one project is shown in \ref{figure12}.
\begin{figure}[H]
	\centering
	\includegraphics[width=1.0\columnwidth]{ProjectOverview_FinishedProject}
	\caption{Project overview for one finished project}
	\label{figure12}
\end{figure}

\subsection{Project time range overview}
To display the chart, the user first has to choose the applicable time range. On hitting the "Filter setzen" button, the time range is set, and a line chart is drawn for every project, showing the development of planned time and tracked time during the selected time range. To reset the filter, the "Filter neu setzen" button has to be hit. This chart allows to see how the time amounts displayed in the other charts have been achieved and thus allows a user to estimate the quality of their planning. An example for one project is shown in \ref{figure11}.
\begin{figure}[H]
	\centering
	\includegraphics[width=1.0\columnwidth]{TimeOverview}
	\caption{Time overview for one project}
	\label{figure13}
\end{figure}

\section{Application status display and synchronization} \label{Status display}
In the original prototype, a Blazor page for synchronization with Toggl had already been added. As described in chapter \ref{JS replacement}, the modal entry mask was replaced by blazor input components (picture \ref{Toggl page initial}). When the "Save Toggl settings" button is clicked, the credentials entered in the text fields are saved using Local Storage (see chapter \ref{Local Storage}), and a first synchronization request is sent to Toggl. If one or both fields are left empty, appropriate error messages are displayed (picture \ref{Toggl page validation}). After successful synchronization, the entry fields and the save button are replacaed by a "Synchronize" button in order to allow users to re-synchronize the ATP with Toggl at any time and not just on entering the Toggl credentials. A label next to the button indicates the time of the last synchronization. Below the button, an overview of the loaded data and their associations with the ATP planning data is displayed, see picture \ref{Toggl associated}. If a Toggl project has no associations yet, this is indicated in the load overview (see picture \ref{Toggl loaded}). Entries which do not belong to a Toggl project are displayed as belonging to "Entries without project" (picture \ref{Toggl no project}). Internally, these entries are saved in a project having this name. Toggl projects which have been deleted in Toggl, but are still present in the ATP, are marked as deleted (picture \ref{Toggl deleted}). They will be present in the ATP until Local Storage is cleared (e. g., when the browser cache is cleared).

\begin{figure}[H]
	\centering
	\includegraphics[width=1.0\columnwidth]{Toggl_init}
	\caption{Toggl page with no Toggl projects loaded and no synchronization done yet}
	\label{Toggl page initial}
\end{figure}

begin{figure}[H]
\centering
\includegraphics[width=1.0\columnwidth]{Toggl_validation}
\caption{Toggl page after hitting "Save Toggl settings" without having entered any credentials}
\label{Toggl page validation}
\end{figure}

begin{figure}[H]
\centering
\includegraphics[width=1.0\columnwidth]{Toggl_associated}
\caption{Toggl page after loading of one Toggl project which is associated to a plan project}
\label{Toggl associated}
\end{figure}

begin{figure}[H]
\centering
\includegraphics[width=1.0\columnwidth]{Toggl_loaded}
\caption{Toggl page after loading of one Toggl project which is not associated to any plan project}
\label{Toggl loaded}
\end{figure}

begin{figure}[H]
\centering
\includegraphics[width=1.0\columnwidth]{Toggl_noProject}
\caption{Toggl page after loading of a Toggl project and some entries which do not belong to a Toggl project}
\label{Toggl no project}
\end{figure}

begin{figure}[H]
\centering
\includegraphics[width=1.0\columnwidth]{Toggl_deleted}
\caption{Toggl page after loading of a Toggl project and }
\label{Toggl deleted}
\end{figure}

\section{Linking of plan projects and Toggl projects} \label{Linking}