%%% Local Variables:
%%% mode: latex
%%% TeX-master: "../doc"
%%% coding: utf-8
%%% End:
% !TEX TS-program = pdflatexmk
% !TEX encoding = UTF-8 Unicode
% !TEX root = ../doc.tex

In this chapter, the approaches and methods applied are discussed. Apart from that, the goals set in chapter \ref{Definition} are reflected, and possible prospects are shown.

\section{Discussion}
In this chapter, the approaches and methods applied during realization of the project are discussed.

\subsection{Use of Blazor and Fluxor}
Since the application should work platform-independently, and other implementation approaches like Java desktop apps or pure web apps were not deemed to be suitable, Blazor was a fitting choice. With mostly concise documentation provided by Microsoft, implementation is made easy. Fluxor, on the other hand, might not be very intuitive at first glance. However, it provides an elegant way of frontend behaviour control as well as a clear separation between synchronous and asynchronous code. In the light of this, the combined use of these two frameworks seems appropriate for the realization of the ATP.

\subsection{Use of Plotly.Blazor}
Whereas the documentation of Plotly.Blazor was rather minimalistic, the documentation of Plotly for other languages like Python was quite extensive and thus allowed for implementing the charts display as needed. Beside that, the amount of code needed for the charts display could be kept at a minimum, compared to other packages providing chart components (see chapter \ref{Charts}). Even more complex use cases like the combination of stacked and grouped bar charts could be simulated by clever arrangement of the functionalities provided. Therefore, Plotly.Blazor is a very suitable choice for this application.

\subsection{Application of reduced Scrum methodology}
Applying the two week sprint approach was especially helpful at the beginning of the project, allowing for more accurate planning of the different tasks. However, as the project went on, the four milestones defined at the beginning became more important than the biweekly sprint plan. What is more, most elements of the Scrum methodology were redundant, since the project team consisted of only two persons and one supervisor. In view of this, Scrum does not seem to be applicable in this kind of project. Milestones would be sufficient and can be divided into sub-milestones, if needed.

\subsection{Time planning}
Generally, the planned schedule (see chapter \ref{Time management}) could be kept. The only exception was with milestone 2 (application status display), where the end was postponed from the Friday to the following Monday due to work for other lectures. The weekly meetings with Mr. Feisthammel were sometimes moved from Tuesday to Wednesday morning due to scheduling issues, which did not impact the progress of the project, as we had lectures until 9.00 p. m. on Tuesdays and therefore, work on the project continued usually on Wednesday morning.

\section{Reflection on project goals}


\section{Prospects}