%%% Local Variables:
%%% mode: latex
%%% TeX-master: "../doc"
%%% coding: utf-8
%%% End:
% !TEX TS-program = pdflatexmk
% !TEX encoding = UTF-8 Unicode
% !TEX root = ../doc.tex

In this chapter the approaches and methods applied are discussed. Apart from that, the goals set in chapter \ref{Definition} are reflected and possible prospects are shown.

\section{Discussion}
In this chapter the approaches and methods applied during realization of the project are discussed.

\subsection{Use of Blazor and Fluxor}
Since the application should work platform-independently and other implementation approaches like Java desktop apps or pure web apps were not deemed to be suitable Blazor seemed a fitting choice. Implementation was made easy with documentation provided by Microsoft, most of it concise. Fluxor, on the other hand, did not appear to be very intuitive at first glance. However, it provides an elegant way of front-end behavior control as well as a clear separation between synchronous and asynchronous code. In the light of this the combined use of these two frameworks was deemed appropriate for the realization of the ATP.

\subsection{Use of Plotly.Blazor}
Whereas the documentation of Plotly.Blazor was rather minimalistic, the documentation of Plotly for other languages like Python was quite extensive and thus allowed for implementing the charts display as needed. Apart from that, the amount of code needed for the charts display could be kept at a minimum compared to other packages providing chart components (see chapter \ref{Charts}). Even more complex use cases like the combination of stacked and grouped bar charts could be simulated by clever arrangement of the functionalities provided. Therefore, Plotly.Blazor is a very suitable choice for this application.

\subsection{Application of reduced Scrum methodology}
Applying the two week sprint approach was especially helpful at the beginning of the project allowing for more accurate planning of the different tasks. However, as the project went on the four milestones defined at the beginning became more important than the biweekly sprint plan. What is more, most elements of the Scrum methodology were redundant since the project team consisted of only two persons and one supervisor. In view of this, Scrum does not seem to be applicable to this kind of project. Milestones would be sufficient and can be divided into sub-milestones, if needed.

\subsection{Time planning}
Generally, the planned schedule (see chapter \ref{Time management}) could be kept. The only exception was with milestone \#3 (application status display) where the end was postponed from the Friday to the following Monday due to work for other lectures. The weekly meetings with Mr. Feisthammel were sometimes moved from Tuesday to Wednesday morning due to scheduling issues; this did not impact the progress of the project as we had lectures until 9.00 p.m. on Tuesdays and therefore work on the project continued usually on Wednesday mornings.

\section{Reflection on project goals}
Compared to the definition of the project (see chapter \ref{Definition}) all project goals defined at the beginning have been fulfilled. The Blazor application runs locally in the browser. Planning data is managed by means of ATP plan projects which can be imported, exported and newly created using the user interface of the application. It is also possible to synchronize the Toggl data by fetching the tracked time data from Toggl Track. The last time of synchronization as well as the status of the loaded data are displayed on the user interface. The difference between planned and tracked time data can be read from the charts.

\section{Prospects} \label{Prospects}
This section describes possible additions and enhancements for the ATP application which might be realized in the future.

\subsection{User experience}
With the standard Blazor input components used so far some parts of the application are awkward to use. For instance, the linking of plan and Toggl projects uses radio buttons instead of checkboxes for the selection of Toggl projects. Therefore, all Toggl projects to be linked to one plan project have to be linked one after another. Moreover, the application looks rather old-fashioned as no CSS rules or other styling improvements have been applied in this prototype. Apart from this, the usability of the application could be enhanced by adding confirmation requests on deletion of plan projects. Since Blazor input components provide very limited functionality for application handling, an additional package providing more user controls (e.g., Blazorise or Radzen, see chapter \ref{Charts}) might be helpful.

\subsection{Error handling}
If an unhandled exception occurs, the application currently displays a yellow error bar at the bottom of the page indicating that an error occurred. At the moment, the only way to figure out the cause of the error is to open the browser console and try to understand the error messages printed there. This is only useful for developers knowing the source code. A more concise error message, possibly providing appropriate options to exit the error state, would significantly enhance the usability of the application.

\subsection{Charts}
Currently, the charts only display data on project level. While this is sufficient for small projects, larger ones might not be displayed with adequate accuracy because the different tasks cannot be viewed individually. This could be solved by adding an additional view allowing the separate display of the tasks of a previously selected project.

\subsection{Cross-platform functionality}
In the original prototype cross-platform functionality was enabled by releasing the software on a docker image which could be run on any platform. However, the GitHub Action that had been implemented did not work anymore for the current prototype. Since this feature was not included into the scope of this project it was not given any further consideration. The software currently only runs on Windows and can be downloaded directly from the GitHub project repository. For future versions of the application it is important that the cross-platform functionality is implemented again. The original GitHub action is left in the project repository and may be used as a base.

\subsection{Integration of other time tracking platforms}
As Toggl Track is not the only time tracking platform available and other platforms might be used by students and teachers as well the integration of such platforms is another enhancement to be taken into account. The existing Toggl page could be extended such that users first would have to select the platform before entering their credentials. On the other hand, different platforms use different data formats for tracked time which would require the ATP to wrap the different data formats into one unified format.

\section{Possible first steps for successors}
The very first step for developers making themselves familiar with the current implementation of the ATP is to gain a fundamental understanding of Blazor and Fluxor. If the use of Plotly is intended to be continued, they should also study the current charts display source code and the documentation of Plotly chart components. Another important aspect is the calculation of the chart data and of the mapping of Toggl projects to plan projects. Possible enhancements and additions to be implemented are described in chapter \ref{Prospects}.