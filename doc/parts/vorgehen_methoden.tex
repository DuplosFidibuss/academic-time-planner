%%% Local Variables:
%%% mode: latex
%%% TeX-master: "../doc"
%%% coding: utf-8
%%% End:
% !TEX TS-program = pdflatexmk
% !TEX encoding = UTF-8 Unicode
% !TEX root = ../doc.tex

This section describes the approaches and methods applied in the project. These consist of the proof that the
approach given by the previous bachelor thesis as described in chapter tbd is applicable, the organisation methodology,
the use of Git version control (see chapter tbd) and the project schedule set up at the beginning of the project.

\section{Proof that approach of bachelor thesis is applicable}
The first step as to realisation of this project was to prove whether the approach given by the bachelor thesis
on which the project is founded is applicable. Since neither one of us has ever worked with Blazor or the 
related technologies used in the bachelor thesis, the consideration of the software prototype was of particular
importance. Both the prototype (see chapter tbd) and the concept of mapping Toggl Track data to ATP data (see
chapter tbd) have been investigated in detail. This was done using the documentation of the bachelor thesis,
the software code of the prototype and the Blazor online documentation. Thus a fundamental understanding of the
approach has been gained. Moreover, our .NET teacher, Prof. Karl Rege, has agreed to provide technological support 
for Blazor and related technologies. In the light of all this, the approach has been considered applicable and
therefore served as basis for the realisation of this project.

\section{Milestones} \label{Milestones}
Four milestones were proposed at the beginning of the project. They were taken from the project outline and include three programming milestones and one documentation milestone. The first one was the implementation of the overview of difference between plan and tracking, followed by the application status display. The final programming milestone was composed of the import and export as well as the creation of plan projects. After this the documentation had to be finalized. These milestones were meant as a guideline to see if the progress was adequate or if we had to rethink our approach. Those milestones were created and managed in GitHub (see chapter \ref{GitHub}). 

